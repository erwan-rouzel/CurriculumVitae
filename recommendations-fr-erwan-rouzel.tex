\documentclass[12pt]{article}

\usepackage[francais]{babel} 
\usepackage[utf8]{inputenc}  %% les accents dans le fichier.tex
\usepackage[T1]{fontenc}       %% Pour la césure des mots accentués
\usepackage[paper=a4paper,textwidth=160mm]{geometry}
\usepackage{graphicx}
\usepackage{extsizes}
\usepackage{wasysym}
\usepackage{url}                % citer des adresses électroniques et des URL
\usepackage{vmargin}            % redéfinir les marges
\setmarginsrb{1cm}{1cm}{1cm}{1cm}{0cm}{0cm}{0cm}{0cm}

\pagenumbering{gobble}

\newif\ifquoteopen
\catcode`\"=\active % lets you define `"` as a macro
\DeclareRobustCommand*{"}{%
   \ifquoteopen
     \quoteopenfalse ''% 
   \else
     \quoteopentrue ``%
   \fi
}

\begin{document}

\begin{center}
\par\textbf{\LARGE Recommandations}\\
\large{Erwan Rouzel - Ingénieur Logiciel}
\end{center}
\vspace{0.5cm}
\noindent \textbf{Eric Salomon} -- \textit{Release Introduction Manager},  Alcatel-Lucent\hfill
\phone\hspace{0.2cm} 06 25 18 11 90
\\\\
2003 -- 2004 (en tant qu'Ingénieur Logiciel):\\
-- Conception et développement d'un outil d'architecture de réseaux mobiles\\
-- Travail en équipe dans un contexte international
\\\\
\noindent "J'ai eu le plaisir d'avoir Erwan dans mon équipe de September 2003 à Juin 2004, en tant que stagiaire pour sa dernière année avant d'être diplômé de l'ENST Bretagne. Erwan a ensuite travaillé à un plein temps pour Alcatel en tant qu'ingénieur en développement logiciel, et a participé au développement d'un outil de gestion d'architecture de réseaux télécom, montrant non seulement de très bonnes compétences en programmation, mais aussi la capacité à intégrer complètement une équipe international répartie à travers la France et la Roumanie. Sa curiosité et sa créativité se sont révélées être un atout décisif pour obtenir la bonne architecture réseau qui était requise à ce moment."

\begin{center}
\line(1,0){250}
\end{center}

\noindent \textbf{Jean-François Pillet} -- \textit{Chef d'entreprise},  Efficience Multimedia\hfill\phone\hspace{0.2cm} 01 44 61 06 06\\\\
2009 -- 2011 (en tant qu'Ingénieur Logiciel - Consultant Indépendant) :\\
-- Conception et développement du site Veolia Environnement\\
-- Gestion de projet pour un site de Réseaux Ferrés de France\\
\\
"J'ai eu l'occasion de travailler avec Erwan sur plusieurs projets complexes en terme de gestion et il a toujours su montrer une grande attention dans le suivi et un état d'esprit positif, qui viennent enrichir son expertise technique."

\begin{center}
\line(1,0){250}
\end{center}

\noindent \textbf{Arnaud Louvet} -- \textit{Chef d'entreprise},  Voyelle\hfill
\phone\hspace{0.2cm} 02 30 96 07 63\\\\
2010 -- 2011 (en tant qu'Ingénieur Logiciel - Consultant Indépendant) :\\
-- Formation technique d'une équipe de développeurs\\
-- Migration des données pour la nouvelle version du Portail de l'Innovation de la Bretagne\\\\
"Erwan nous a accompagné sur un projet d'ampleur sur des actions de formation et de développement sur la technologie eZ Publish. 
Tant sur la pédagogie que le développement il maitrisait son sujet."

\begin{center}
\line(1,0){250}
\end{center}

\noindent \textbf{Alain Mathieu} -- \textit{Chef d'entreprise},  Brain\hfill
\phone\hspace{0.2cm} 06 60 70 96 27
\\\\
2011 -- 2012 (en tant qu'Ingénieur Logiciel - Consultant Indépendant) :\\
-- Migration du site apec.fr vers la nouvelle version\\
-- Conception et développement de l'outil d'import / export
\\\\
"J'ai été ravi de travailler avec Erwan qui est intervenu pour notre client l'APEC dans le cadre d'une migration dans un environnement complexe."

\begin{center}
\line(1,0){250}
\end{center}

\end{document}